\documentclass[15pt,a4paper]{article}
\usepackage[utf8]{inputenc}
\usepackage{amssymb}
\usepackage{polski}
\usepackage{color}
\title{Zadanie Domowe}
\author{Mateusz Orłowski gr 3ID15A}
\begin{document}
\maketitle


\begin{center}
{\huge Zadanie z czcionką  : \\}


\begin{HUGE}
\textcolor{white}{Mateusz Orłowski Zadanie 1}\\
\end{HUGE}


\begin{Large}
\textcolor{red}{Mateusz Orłowski Zadanie 1}\\
\end{Large}



\begin{huge}
\textcolor{blue}{Mateusz Orłowski Zadanie 1}\\
\end{huge}

\pagecolor{green}

{\Huge Tabela\\}
\begin{center}
\begin{tabular}{|c|c|} \hline
Politechnika Świętokrzyska & Kielce \\
\hline \hline
Temat & Zadanie domowe w Latex \\
\hline \hline
Laboratorium & 1 \\
\hline \hline
Grupa & 3ID15A \\
\hline \hline
Przedmiot & IOT \\ 
\hline \hline
Data wykonania & 24.10.2018 \\
\hline \hline
\end{tabular}
\end{center}
\section{BIBLIOGRAFIA}
\end{center}





\cite{abramowitz+stegun}
\bibliographystyle{ieeetr}
\bibliography{references}
\bibliographystyle{ieeetr}
\newpage
\begin{center}
\section{Wnioski}
\end{center}

LaTeX – oprogramowanie do zautomatyzowanego składu tekstu, a także związany z nim język znaczników, służący do formatowania dokumentów tekstowych i tekstowo-graficznych (na przykład: broszur, artykułów, książek, plakatów, prezentacji, a nawet stron HTML)

Tworzenie tekstu w LaTeX-u opiera się na zasadzie WYSIWYM (What You See Is What You Mean - To, co widzisz, jest tym, o czym myślisz). Od zasady WYSIWYG odróżnia go to, że autor tekstu określa jedynie logiczną strukturę dokumentu (tzn. zaznacza, gdzie zaczyna się rozdział, co jest przypisem itp.), natomiast samym graficznym "ułożeniem" tekstu na stronie zajmuje się TeX, zwalniając tym samym użytkownika z tego zadania.

\end{document}
\end{document}




